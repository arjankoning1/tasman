\begin{samplecase}
{\bf n-Nb093-omp-opt: Optimization of optical model parameters for total cross section}\newline
The sample case below does a simple optical model search, on total cross sections only.
The {\it tasman.inp} file looks as follows

\VerbatimInput{\samples n-Nb093-omp-opt/org/tasman.inp}

and you can run this after going to {\it /samples/n-Nb093-omp-opt/new/} with {\it tasman < tasman.inp > tasman.out \&}.
TASMAN does not have the same flexibilty for the {\bf energy} keyword as TALYS, so you always have to give either 1 incident energy or
your own list of incident energies, like in TALYS. (for example, {\bf energy 0.5 20 0.5} does not work for TASMAN.)
Note that for this sample case, we stop at a maximum of 100 TALYS runs, and you may need a lot more to reach convergence.
With the {\bf \#expinclude} keyword we automatically include all experimental
data from the EXFOR database for MT1, i.e. total
cross sections. We limit the search to 9 different parameters and for this particular sample case we give them explicitly including their
starting values, so TASMAN knows how to change them.
Each time the search method finds a new optimum, it is mentioned in the main
output file {\it tasman.out}.

The main output file now contains more information. Below the used parameters
for the particular random input file there is now a block with a goodness-of-fit summary of the TALYS run, when compared with experimental data,
{\small \begin{verbatim}
 Comparison of TALYS and experimental data for  Z:  41 A:  93  Run:     0

 Goodness-of-fit summary for Z:  41 A:  93

 Channel         Sets  Points     GOF           Total optimum         Channel optimum
totalxs.tot        25   866      1.115         1.0000E+38 (   0)     1.0000E+38 (   0)

GOF no.                0   1.115
Number of nuclides     1
Number of channels     1
Number of data sets   25
Number of points     866                Optimum
Total chi2             271.0
chi2 per point         10.84            1.0000E+38
F per point           1.0239E+06        1.0000E+38
Total D               2.5596E+07
D per point           1.0239E+06        1.0000E+38
Av. distance           1012.     mb     1.0000E+19 mb
K per point            1.271            1.0000E+38

 New optimum G=    1.11451721
\end{verbatim} } \renewcommand{\baselinestretch}{1.07}\small\normalsize
\noindent
In our case, the optimum is found for sample run number 95 which gives

{\small \begin{verbatim}
 New optimum G=  1.10180080
\end{verbatim} } \renewcommand{\baselinestretch}{1.07}\small\normalsize
\noindent
The block above is just a summary, giving the number of sets, and various
goodness-of-fit estimators, per channel, and as total. This is given for every random run. Run number 0 is always the one with default parameter values.
Note that the default is to use the F-value as the leading goodness-of-fit
estimator. This can be changed with the {\bf \#chi2} keyword.
More extensive information about the goodness-of-fit is given in the {\it gof.NNNN} files where NNNN denotes the random run. You may want to study e.g. {\it gof.0001} which gives details about every included experimental data set, and point, and its deviation from the TALYS calculation. At the bottom of the file,
the summary for each particular random run is given, just like in the main
output file. Note also that inside {\it gof.NNNN} averages per reaction channel are given.
Moreover, when a new optimum is found, the current optimal set of TALYS output files and parameters are copied to a directory
{\it optimum/}. In the process, some linux 'cp' warnings may occur for non-existing files, as I was not able to get rid of them.
\end{samplecase}

