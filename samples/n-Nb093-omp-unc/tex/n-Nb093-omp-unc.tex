\begin{samplecase}
{\bf n + ${}^{93}$Nb: Basic, unweighted uncertainty calculation}\newline

The first sample case is given below. It concerns a default calculation for
neutrons on $^{93}$Nb, with an energy grid specified in the file
{\it energies}. As explained in this manual, the {\bf \#change} keyword is
essential.
After this line, TASMAN keywords, always starting with a '\#', can be given
while comment lines after {\bf \#change} need to start with '\#\#'.
Keyword {\bf \#mode 1} specifies that we are interested in an uncertainty
calculation, i.e. a statistical spread of the final results.
{\bf \#band y} means that the final cross section files contain, besides
the central values, the lower and upper 1-sigma uncertainty band.
In this sample case, only 2 parameters are varied, namely the radius and
diffuseness of the real volume central OMP. Since only the central value,
for both parameters 1., is specified, the default uncertainties for these
parameters are used.

\VerbatimInput{\samples n-Nb093-omp-unc/org/tasman.inp}

This input file called {\em tasman.inp} can simply be run as follows:\newline

{\bf tasman $<$ tasman.inp $>$ tasman.out}\newline

The {\it tasman.out} file starts with a display of the version of TASMAN
you are using,
the name of the author, and the Copyright statement. Also the physics
dimensions used in the output are given.
As can be seen in the output below, the first section of the output is a print of the keywords/input
parameters. This is done in two steps: In the first block
an exact copy of the input file as given by the user is returned. In
the next block
a table with all keywords is given,
not only the ones that you have specified in the input file, but also all the
defaults that are set automatically. The corresponding Fortran variables are
also printed, together with a short explanation of their meaning. This table
can be helpful as a guide to change further input parameters for a next run.
You may also copy and paste the block directly into your next input file.

In the next output block
we print the random parameters used at each run.
At the end of this simple output, the total calculation time is printed, followed by a
message that the calculation has been successfully completed.

\VerbatimInput{\samples n-Nb093-omp-unc/org/tasman.out80}

More interesting are the various output files that have been generated.
Apart from the standard output files of TALYS, like {\it totalxs.tot},
{\it xs200000.tot}, etc. you will also find various files
whose names end with {\it .ave}. These are the average cross section files,
including an uncertainty band.
These results are obtained after 10 random TALYS calculations.

A full covariance matrix, correlating all channels and energies with each other, is available. 
For example in {\em cov\_intra.ave} we have:

{\small \begin{verbatim}

# header:
#   title: Nb93(n,tot) cross section
#   source: TASMAN
#   user: Arjan Koning
#   date: 2023-12-16
#   format: YANDF-0.1
# target:
#   Z: 41
#   A: 93
#   nuclide: Nb93
# reaction:
#   type: (n,tot)
#   ENDF_MF: 3
#   ENDF_MT: 1
# covariance:
#   type: in-channel covariance
# datablock:
#   quantity: cross section
#   columns: 5
#   entries: 100
##      E_a            E_b           Rcov           Ccov           Vcov
##     [MeV]          [MeV]           []             []             []
   1.000000E-01   1.000000E-01   9.706488E-03   1.000000E+00   8.987610E+05
   1.000000E-01   2.000000E-01   9.188086E-03   9.894124E-01   9.013705E+05
   1.000000E-01   5.000000E-01   6.335274E-03   9.482473E-01   5.574611E+05
   1.000000E-01   1.000000E+00   5.746896E-03   9.581335E-01   3.820715E+05
   1.000000E-01   2.000000E+00   4.502018E-03   9.734957E-01   2.111240E+05
   1.000000E-01   3.000000E+00   1.400545E-03   9.506426E-01   5.472231E+04
   1.000000E-01   4.000000E+00  -5.533088E-04  -8.405757E-01  -1.970543E+04
......................................
\end{verbatim} } \renewcommand{\baselinestretch}{1.07}\small\normalsize
\noindent

%Fig. \ref{unc} shows the difference between the files {\it totalxs.tot},
%which represents the default TALYS run, while the other 3 curves come
%from {\it totalxs.ave} which contains the uncertainty band. Clearly 10 random
%runs is not enough to get a converged result for the average, although it can
%not always be expected that the average after many more runs goes to the
%default run, due to non-linearities in the model.

\end{samplecase}
